\documentclass[letterpaper,10pt]{article}
\usepackage{nuvem}


\begin{document}

\title{Analytic consideration of }

\author{Marilia M. Pisani$^1$\\
Renato Fabbri$^2$}
\address{$^1$CCNH/UFABC, $^2$IFSC/USP}
\email{$^1$marilia.m.pisani@gmail.com, $^2$renato.fabbri@gmail.com}

\begin{abstract}
Social participation had been growing on the last decade
and had a drawback with the recent rise of conservative political parties 
in various locations of the world.
We aim to present a set of developments on linked social data
and analysis over the last years, with a focus
on the potential use for the civil society and scientific academy.
Various conceptualizations have been gathered by interviewing specialists
and State authorities and have been validated by them and their institutions.
Data from known social networks, such as Facebook, Twitter, IRC and Email,
and from more specialized social participation platforms, have been
	translated to RDF and linked to these conceptualizations in the form of OWL ontologies~\cite{losd,pnud5}.
Social networks were reported as very stable and their language varies
	with connectivity~\cite{tese,stab}.
	Also, resources recommendation and experiments have been performed with such data.
	How are we to articulate the gathering and analysis 
	of such data with the needs of the civil society and the academy?
	Will it sufice to deal with the private and State interests
	that shape our society?
	Does Anthropological Physics yield reasoable strategies
	to collect and analyze data from our society?
\end{abstract}

\section{Introduction}
Social analytics by Schmidt, with its practice of reporting tendencies of the times.
But in contrast, it seems reasonable to put emphasys on issues and analytical outcomes
that are very controversial.


Analítica social. sociedade pela sociedade 
  - Consideração da sociedade pela sociedade:
    * O que fizemos até aqui
    * Qual a questão e contexto atual: com as tecnologias,
    é importante a interface com o ferramental para análise dos dados,
    até para fazer frente aos outros poderes.
    * Como podemos trabalhar o tema: conexão com Nuvem, com Nexos e com IFSC/ICMC.

The appearance of this document reflects the requirements of the style guide.  Since there is no typesetting or copy-editing of summaries, the use of this style guide is critical to provide a consistent appearance.

The first line of the first paragraph of a section or subsection should start flush left. The first line of subsequent paragraphs within the section or subsection should be indented 0.2 in. (0.62 cm).

Use 8.5 in. x 11 in. paper (21.505 cm x 27.83 cm) with 1 in. margins (2.54 cm) on all sides, use 10-point Times New Roman or Palatino font, and do not use hyphens at the end of a line. 

\section{What has been done}

\subsection{Social participation data}
\subsection{Ontologies}
\subsection{Critical theory}
\subsection{Fundamental cycle}

\section{What shall we do?}
% keep the data online
% develop an audiovisual analytics software
% maintain partnerships between humanities and hard sciences

\section{Conclusions}

\begin{thebibliography}{99}

\bibitem{krishnan00} E. Krishnan, A. M. Shan, T. Rishi, L. A. Ajith, C. V.
Radhakrishnan, \textit{On-line Tutorial on \LaTeX{}},
``Mathematics'' (Indian \TeX{} Users Group, 2000), \\
\url{http://www.tug.org/tutorials/tugindia/chap11-scr.pdf}.

\bibitem{vantrigt97} C. van Trigt, ``Visual system-response functions and estimating reflectance,''
J. Opt. Soc. Am. A \textbf{14}, 741--755 (1997).

\bibitem{masters93} T. Masters, \emph{Practical Neural Network Recipes in C++} (Academic, 1993).

\bibitem{shoop97} B. L. Shoop, A. H. Sayles, and D. M. Litynski, ``New devices for optoelectronics: smart pixels,''
in \emph{Handbook of Fiber Optic Data Communications},
C. DeCusatis, D. Clement, E. Maass, and R. Lasky, eds. (Academic, 1997), pp. 705--758.

\bibitem{kalman76} R. E. Kalman,``Algebraic aspects of the generalized inverse of a rectangular matrix,'' in
\emph{Proceedings of Advanced Seminar on Generalized Inverse and Applications}, M. Z. Nashed, ed. (Academic, 1976), pp. 111--124.

\bibitem{craig96} R. Craig and B. Gignac, ``High-power 980-nm pump lasers,''
in \emph{Optical Fiber Communication Conference}, Vol. 2 of 1996 OSA Technical Digest Series (Optical Society of America, 1996), paper ThG1.

\bibitem{steup96} D. Steup and J. Weinzierl, ``Resonant THz-meshes,''
presented at the Fourth International Workshop on THz Electronics, Erlangen-Tennenlohe, Germany, 5--6 Sept. 1996.

\end{thebibliography}


\end{document}
