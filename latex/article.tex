\documentclass[12pt,fleqn]{article}
\usepackage{xiiiemc}
\usepackage{natbib}
\usepackage{fancyhdr}
\usepackage{fancyvrb}
\usepackage{color}
\usepackage{wallpaper} 
\usepackage{titlesec}   %% Define space between paragraph e section
\usepackage{float} 	%% Use to fix Figure or Table: ex: \begin{table}[H]
\usepackage[usenames,dvipsnames]{xcolor}
\usepackage{hyperref} 	%% Use to fix Figure or Table: ex: \begin{table}[H]
\hypersetup{
	%pagebackref=true,
	pdfcreator={LaTeX with abnTeX2},
	pdfkeywords={abnt}{latex}{abntex}{USPSC}{trabalho acadêmico}, 
	colorlinks=true,       		% false: boxed links; true: colored links
	linkcolor=blue,          	% color of internal links
	citecolor=blue,        		% color of links to bibliography
	filecolor=magenta,      		% color of file links
	urlcolor=blue,
	allbordercolors=black,
	bookmarksdepth=4
}
\usepackage{tocloft}
\titleformat{\section}
  {\normalfont\bfseries}{\thesection.}{0.5em}{}
\renewcommand\cftsecaftersnum{.} 
\renewcommand\thesection{\arabic{section}}
\renewcommand\thesubsection{\thesection.\arabic{subsection}}
%%%%Don't edit this block. It reduces the spacing between the lines of the references
\let\OLDthebibliography\thebibliography
\renewcommand\thebibliography[1]{\OLDthebibliography{#1} \setlength{\parskip}{0pt}\setlength{\itemsep}{0pt plus 0.3ex}}

\usepackage{grffile}
\usepackage{afterpage}

%%-----------------------------------------------EDIT-----------------------------------------------
\title{Audiovisual Analytics Vocabulary and Ontology (AAVO): initial core and example expansion}

%%-----------------------------------------------EDIT----------------------------------------------
\author
    {\rm \begin{tabular}{l} 
    \textbf{Renato Fabbri}$$ - {\textnormal renato.fabbri@gmail.com}\\%
    \textbf{Maria Cristina Ferreira de Oliveira}$$ - {\textnormal cristina@icmc.usp.br}\\
    {\fontsize{11}{0}\selectfont University of São Paulo, Institute of Mathematical and Computer Sciences - São Carlos, SP, Brazil}\vspace*{-0.05cm} \\
%    {\fontsize{11}{0}\selectfont $^{2}$Federal University of ABC, Centre for Natural Sciences and Humanities - São Paulo, SP, Brazil}\vspace*{-0.05cm}\\
  \end{tabular}}
%%----------------------------------------------------------------------------------------------

\fancypagestyle{firspagetstyle}
{
	\lhead{}
	\fancyhead[C]{%
		\includegraphics[width=0.9\linewidth]{logo}\\%
		{\scriptsize \fontfamily{phv}\fontseries{b}\selectfont \color[rgb]{0.45,0.45,0.45}
		16 a 19 de Outubro de 2017\\
		Instituto Politécnico - Universidade do Estado de Rio de Janeiro\\
		Nova Friburgo - RJ\\
	    }
	}
	\renewcommand{\headrulewidth}{0.0pt}
	\fancyfoot[C]{\footnotesize \parbox{15cm} {\centering  \fontsize{7.5}{0}\selectfont \it Anais do XX ENMC – Encontro Nacional de Modelagem Computacional e VIII ECTM – Encontro de Ciências e Tecnologia de Materiais,  Nova Friburgo, RJ – 16 a 19 Outubro 2017}} % \ttfamil
	\rhead{}
}


\begin{document}
\maketitle

\thispagestyle{firspagetstyle}

\fancyhead[L]{\footnotesize{\fontsize{7.5}{0}\selectfont \it XX ENMC e VIII ECTM\\
	16 a 19 de Outubro de 2017\\
	Instituto Politécnico Universidade do Estado do Rio de Janeiro – Nova Friburgo - RJ\\}}
\renewcommand{\headrulewidth}{0.0pt}
\fancyfoot[C]{\footnotesize \parbox{15cm} {\centering  \fontsize{7.5}{0}\selectfont \it Anais do XX ENMC – Encontro Nacional de Modelagem Computacional e VIII ECTM – Encontro de Ciências e Tecnologia de Materiais,  Nova Friburgo, RJ – 16 a 19 Outubro 2017}} % \ttfamil
\rhead{}

\begin{abstract}
Visual Analytics might be defined as data mining though interactive visual interfaces.
The field and has received prominent consideration by researchers, developers and the industry. 
The literature, however, is complex because it involves many fields of knowledge
and is considerably recent.
In this article we describe an initial organization of this knowledge as an OWL ontology
and a SKOS vocabulary.
This effort might be useful in many ways that include conceptual considerations and
software implementations.
Within the results and discussions, we expose a core and an example expansion
of the conceptualization, and incorporate design issues that enhance the
expressive power of the abstraction. 
\end{abstract}

\keywords{\em{OWL, SKOS, Semantic web, Visual analytics, Data visualization}}

\pagestyle{fancy}

\section{INTRODUCTION}\label{sec:intro}
Visual Analytics' usual definition is data mining (or the science of analytical reasoning facilitated)
through interactive visual interfaces.
From this definition, one can grasp that at least three fields are directly related to visual analytics:
data mining, human-computer interaction, and data visualization.
Each one of these fields are multidisciplinary and known to be considerably complex
with multiple theories and vast literature.

This work reports an initial organization of the knowledge related to visual analytics
as a SKOS vocabulary and an OWL ontology, i.e. a formalized conceptualization
in terms of linked data recommended technologies.
The uses this formalization can have include:
\begin{itemize}
	\item introduction of the Visual Analytics subject for non-specialists.
	\item Expressing a concise overview.
	\item Queries.
	\item Relating objects (e.g. data and techniques) inside the field or between the field and other domains.
	\item Making inference about the concepts and objects to which they are related.
\end{itemize}

In the present case, where the formalization is realized as linked data,
the conceptualization allows the queries, inferences, relations, etc.
to be performed also by machines.
Therefore, for example, a software might relate a dataset
or analysis methods to visualization techniques in order to assist a user or
for automated reporting.

Section~\ref{sec:methods} uncovers the methods and technologies
needed for stating and discussing the results in Section~\ref{sec:res}.
Final remarks and future work are in Section~\ref{sec:con}.

\section{METHODS}\label{sec:methods}
The methods described here are standard of the semantic web
to achieve formalized conceptualizations.
Thus, the next sections address the subjects very briefly.
The interested reader should visit the vanilla literature,
especially the W3C recommendations~\citep{w3cld,ldb}.

\subsection{The semantic web}
The semantic web is constituted by data which is linked in the same way
web pages are: through HTTP and URLs.
W3C recommendations are the main sources of protocols and best practices of the field.
The terms 'linked data' and 'semantic web' are most often used interchangeably.
A distinction might arise in some contexts where one needs to refer to the linked data or
the semantic web created by all or some portion of linked data, but, as a knowledge and technological
field, the terms are equivalent.
The main topics of this article are data visualization (or visual analytics)
and semantic web, and all the sections tackle the subject of the semantic web
for it is the framework in which data visualization knowledge is represented.

\subsection{RDF}
The semantic web is built using the Resource Description Framework (RDF).
The RDF data model is based on making statements in the form of triples
(``subject-predicate-object'') and using Unique Resource Identifiers (URIs) for
objects and concepts.
It is also part of the framework to use URIs that are URLs whenever possible,
to enable the data linkage.
Accordingly, one can write:

\begin{Verbatim}[fontsize=\footnotesize]
<http://example.org/people/mary>
	<http://example.org/properties/name> "Mary Shastacian" .
<http://example.org/people/mary>
	<http://example.org/properties/age> "57" .
<http://example.org/people/mary> 
	<http://example.org/properties/likes> 
                <http://example.org/concepts/Reading> .
\end{Verbatim}
\noindent to express that there is a 57 years old person called Mary Shastacian which likes reading.
There are many formats to write/serialize RDF data.
The example above is written in Turtle and it will be the format used throughout this document.

In real settings, when everything is working as recommended,
each of these URIs (that are URLs!) might be accessed 
through HTTP to reach more triples referring to the URI.
From the triples above, one would be able to access triples
describing each of the properties: \texttt{example:properties/name}, \texttt{example:properties/age}, and\\
\texttt{example:properties/likes},
and the concept \texttt{example:concepts/Reading}.
In fact, the triples above would probably be available in the URI/URL:
\texttt{example:people/mary}.
One good working example\footnote{Visit \url{http://dbpedia.org/page/Rhesus_macaque}
and click on the concepts and properties to start browsing the web of linked data.}
of this is DBPedia~\citep{dbpedia}.
The process of accessing a URI to find more triples is called \emph{dereferencing the URI} (or simply dereferencing).

\subsection{RDFS}
The Resource Description Framework Schema (RDF Schema or simply RDFS)
is a set of classes and properties for the RDF data model that allows
basic descriptions of ontologies.
It supports taxonomic relations (hypernymy, i.e. relations stating that a concept is more general than another),
bindings of properties to objects and datatypes, and notes (label, comment, see also, is defined by).

\subsection{SKOS}
The Simple Knowledge Organization System (SKOS)
is a data model for representing controlled vocabularies.
SKOS is a W3C recommendation to facilitate publication
and use of vocabularies and is built upon RDF and RDFS.
It is itself a vocabulary for concepts, notation, documentation,
semantic and mapping relations, and collections.

\subsection{OWL}
The Web Ontology Language (OWL) is a language for publishing ontologies on the web.
While RDFS holds basic relations necessary even for very rudimentary organization
knowledge and data, OWL is complex and allows one to formalize elaborate conceptualizations.
Using OWL, an ontology might have properties that are required to satisfy a number or axioms,
and classes that obey restrictions or e.g. are the result of the union of other classes.

\subsection{Interviews with specialists and literature consultation}
The standard approach to design an ontology, according to the literature,
is to interview specialists of the field to which the ontology is related,
or to absorb the established literature, or both.
This work is being developed using both approaches.
The second author is a data visualization specialist which was
interviewed by the first author.
Also, the first author is acquiring a deeper knowledge in the field.

\section{RESULTS AND DISCUSSION}\label{sec:res}
% a word about AAVO, audiovisual, discerning core and expansions
% main concepts
Using the framework exposed in the previous section,
we elaborated an initial vocabulary and ontology for
visual analytics: the AAVO (Audiovisual Analytics Vocabulary and Ontology).
The inclusion of ``audio'' is both a reminder of the possibilities
available for using audio to represent data and perceive patterns,
and a desirable incorporation of audio to visual analytics given
audiovisual capabilites of current ordinary computers.

The main concepts and their interrelations are presented in
Section~\ref{sec:core} while an example extension is on Section~\ref{sec:ext}.
Section~\ref{sec:voc} holds annotations for the vocabulary which are not promptly
given by the previous sections.
Some of the relations bellow are expressed using very recent 
techniques that are described in~\cite{enhance}.
Their meaning, though, might be easily inferred.

\subsection{AAVO core}\label{sec:core}
The core of AAVO is designed to be minimal and hold the following concepts as
depicted in Figure~\ref{fig:minimum}:
\begin{itemize}
	\item Visualization: a technique to generate a Visual Representation from Data.
	\item Visual Representation: a representation of Data by visual cues.
		A Visual Representation can be an Image or an Animation.
	\item Data: a set of values, be them qualitative or quantitative~\citep{wikipData}.
	\item Dataset Type: a type of organization and meaning of data~\citep{munzner}.
	\item Processing: transforms Data into Data. Pre-processing is a kind of Processing.
\end{itemize}

\begin{figure}[!htbp] %h or !htbp
\vspace{-2pt}
\begin{center}
% \includegraphics[height=6.7cm,width=9cm]{../figs/aavo0.01_minimum.png}
\includegraphics[width=\textwidth]{../figs/aavo0.01_minimum.png}
	\caption{AAVO core, discussed in Section~\ref{sec:core}.
	A Visualization outputs a Visual Representation that can be an Image or an Animation.
	A Visualization is suitable for a Dataset Type and for a Processing routine which transforms
	Data into Data.}
\label{fig:minimum}%
\end{center}
\end{figure}

We envision that there should be at least the following concepts in
AAVO core when it reaches maturity:
\begin{itemize}
	\item Hypothesis: a proposed explanation for a phenomenon that might be
		1) given beforehand and amenable to being proved or refuted by an Analysis,
		2) shared by means of an Analysis,
		or 3) presented by means of a Visualization.
	\item Analysis: a set of procedures used to obtain understanding about
		Data or a phenomenon.
	\item Task/Purpose/Application: the goal or objective of an Analysis.
\end{itemize}
\noindent These concepts were not included in AAVO core (e.g. Figure~\ref{fig:minimum})
because we still deciding on the best way to do so.
From the definitions above, raises the question:
should we also include Phenomenon among these core concepts?

Other relations that can be added to the core (or to an extension, but
are directly related to the core):
\begin{itemize}
	\item Visualization is a type of Processing.
	\item Visual Representation is a Dataset Type.
	\item Processing ``suitable for'' Data.
	\item Visualization ``number of dimensions'' real (not double as stated for now).
\end{itemize}

An example of question still left unanswered:
a Visualization only outputs Visual Representation
or can it output other Data(set Type)?
This and many other questions might have a resolution
that are genuinely dependent on the conceptual design of the ontology.

% potential enhancements, from the email to cristina

\subsection{AAVO example expansion}\label{sec:ext}
There are many ways in which the AAVO core might be expanded.
Figure~\ref{fig:exemp} is an example expansion.
Concepts were added which are hyponyms to Dataset Type (Temporal Series, Relational Data),
to Pre-Processing (Z-Score, Cleaning), Processing (MDS, Statistical Test)
and Visualization (Heat Map, Histogram, Scatter Plot, Timeline).
Some examples of further subclasses are also added.
A different kind of expansion was achieved by including (Data) Availability
and the less general concepts of Dynamic Availability and Static Availability.
A Graph is regarded as a bare Network without any context or further attributes beyond
nodes and edges.

\begin{figure}[!htbp] %h or !htbp
\vspace{-2pt}
\begin{center}
\includegraphics[width=\textwidth]{../figs/aavo0.01_exemplified.png}
    \caption{Example incorporation of less general concepts to AAVO: Statistical Test, MDS, Timeline, Z-Score, Network, etc.
	A thorough consideration of this expansion of the AAVO core is at Section~\ref{sec:ext}.
	In case this image is not being properly visualized in paper,
	an online PDF should be at: \url{https://github.com/ttm/aavo/raw/master/latex/article.pdf}.}
\label{fig:exemp}%
\end{center}
\end{figure}

Ideally, AAVO expansions should reach related fields, such as HCI,
by linking to other existing ontologies (such as DBPedia) or incorporating
enough concepts to then bind and rely in third party conceptualizations.

\subsection{Vocabulary annotations}\label{sec:voc}
Beyond what is made explicit in the previous sections,
there are some aspects of the knowledge and language that
are to be directly added to the SKOS vocabulary.
Examples:
\begin{itemize}
	\item in a dataset, an element is also called: an item, an observation, an individual, a point,
		and even a data point and a data row. 
	\item A graph node is also called: a vertex, and every name that are used to designate an element.
	\item A graph edge is also called: a link, a bond, a line, and a connection.
	\item Z-scores are also called: standard scores, normal scores, standardized variables, and z-values. 
\end{itemize}

\section{CONCLUSIONS AND FURTHER WORK}\label{sec:con}
This initial formalized conceptualization of the AAVO
holds some relations which are not explicitly described by
current literature mainly because of the purposes:
1) of reaching a sound
conceptualization that allows a formalization as linked data;
2) of representing the knowledge in Visual Analytics
to enable inference by machines.
There are other uses for AAVO, uncovered in Section~\ref{sec:intro},
for which there are conceptual models available~\citep{munzner,ward}.

Potential further steps include:
\begin{itemize}
	\item the inclusion of the concepts Hypothesis, Analysis, and Task
		into the AAVO core.
	\item Realizing AAVO expansions until the reached concepts can be linked
		to other ontologies that are sound, used and maintained.
	\item Using AAVO for obtaining interesting relations by means of automated
		inference and for assisting a (audio)visual analytics software.
\end{itemize}



\subsection*{\textit{Acknowledgements}}
The authors thank CNPq and FAPESP for the funding received while researching the topic of this article,
the researchers of IFSC/USP and ICMC/USP for the recurrent collaboration in every situation
where we needed directions for investigation.

% ------------------------------------------------------------------------
\begin{thebibliography}{99}
\fontsize{11}{0}\selectfont
\bibitem[Fabbri, 2017]{enhance}
	Fabbri, R. (2017). Enhancements of linked data expressiveness for ontologies.
		Encontro Nacional de Modelagem Computacional 2017 (XX ENMC).
		From \url{https://github.com/ttm/ontologyEnhancements/raw/master/article.pdf}

\bibitem[Heath \& Bizer, 2011]{ldb}
	Heath, T. \& Bizer, C. (2011). Linked Data: Evolving the Web into a Global Data Space (1st edition). Synthesis Lectures on the Semantic Web: Theory and Technology, 1:1, 1-136. Morgan \& Claypool.

\bibitem[Lehmann et al., 2015]{dbpedia}
	Lehmann, J., Isele, R., Jakob, M., Jentzsch, A., Kontokostas, D., Mendes, P. N., ... \& Bizer, C. (2015). DBpedia–a large-scale, multilingual knowledge base extracted from Wikipedia. Semantic Web, 6(2), 167-195.

\bibitem[Munzner, 2014]{munzner}
	Munzner, T. (2014). Visualization analysis and design. CRC press.

\bibitem[W3C, 2010]{w3cld}
	W3C (2017). LINKED DATA CURRENT STATUS, from \url{https://www.w3.org/standards/techs/linkeddata}

\bibitem[Ward et al., 2010]{ward}
	Ward, M. O., Grinstein, G., \& Keim, D. (2010). Interactive data visualization: foundations, techniques, and applications. CRC Press.

\bibitem[Wikipedia, 2017]{wikipData}
	Data. (2017, August 21). In Wikipedia, The Free Encyclopedia. Retrieved
		22:31, August 21, 2017
		, from \url{https://en.wikipedia.org/w/index.php?title=Data&oldid=796493851}

\end{thebibliography}
% ------------------------------------------------------------------------

%For papers written in Portuguese or Spanish.

%\begin{center}
%  TITLE IN ENGLISH
%\end{center}

%\def\abstractname{Abstract}%

%\begin{abstract}
%   Abstract in english
%\end{abstract}

%\keywords{\em{Keywords in english}}

\end{document}
